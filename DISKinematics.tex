\documentclass[12pt]{article}
\usepackage{amsmath,amssymb}
\usepackage{graphicx}

\begin{document}

\section{Deep Inelastic Scattering Kinematics}

\subsection{Conventions and Basic Relations}

We choose the direction of the hadron beam as the $+z$ direction, with
the lepton beam traveling in the opposite direction. The initial
hadron and electron energies are $E_{\rm h}$ and $E_{\ell}$,
respectively. The scattering angle of the lepton is denoted by
$\theta$ ($\theta = \pi$ for the initial lepton direction), and the angle
of the final state hadronic system is denoted by $\gamma$ ($\gamma=0$
for the initial hadron direction).
We denote the momentum four-vectors of the initial
lepton and the  inital hadron as $k$ and $P$, respectively, and the
scattered four-momenta as $k'$ and $P'$. The final state energies of
the lepton and hadronic system are denote $E^\prime_\ell$ and
$E^\prime_{\rm h}$, respectively.


The squared center-of-mass energy, ${s}$, is given by:
%
\begin{equation}
s = 4E_{\ell}E_{\rm h}
%
\end{equation}

The negative squared four-momentum transferred in deep inelastic
scattering (DIS) between the lepton and hadron is
%
\begin{equation}
Q^2 \equiv - q^2 = -(k - k')^2  = s x y
\label{eq:q2xy}
\end{equation}
%
where $x$ is the Bjorken scaling variable and $y$ is the
fractional energy transfer from the lepton, defined as:
%
\begin{equation}
  x = {Q^2 \over 2 P\cdot q} 
\end{equation}
\begin{equation}
  y = {P\cdot q \over P\cdot k}
\end{equation}
%


\subsection{Lepton Variables}

The DIS variables using the scattered lepton variables are given by:
%
\begin{equation}
Q^2 = 2 E_\ell E^\prime_\ell\, (1+ \cos \theta)
\label{eq:Q2e}
\end{equation}
\begin{equation}
y = 1 - {E^\prime_\ell \over 2 E_\ell} \, (1-\cos\theta)
\label{eq:ye}
\end{equation}
%
The variable $x$ can be found from Eq.(\ref{eq:q2xy}).

An expression for $Q^2$ as a function of $x$ based on the scattered
lepton energy $E^\prime_\ell$ can be found by solving for $\cos\theta$
in Eq.(\ref{eq:Q2e}) and plugging into Eq.(\ref{eq:ye}), then using
Eq.(\ref{eq:q2xy}):
%
\begin{equation}
Q^2 = {1- E^\prime_\ell/E_\ell \over {1\over s x} - {1 \over 4E_\ell^2} } 
\label{eq:q2eep}
\end{equation}
%
Likewise, an expression for $Q^2$ as a function of $x$ based on the
scattered lepton angle $\theta$ can be found by solving for $E^\prime_\ell$ in
Eq.(\ref{eq:Q2e}) and plugging into plugging into Eq.(\ref{eq:ye}), then using
Eq.(\ref{eq:q2xy}):
%
\begin{equation}
Q^2 = {s \over {1\over x} + {E_{\rm h} \over E_\ell}({1-\cos\theta
    \over 1+\cos\theta}) } =
{sx \over {1} + x\,{E_{\rm h} \over E_\ell}\,\tan^2({\theta \over 2})} 
\label{eq:q2etheta}
\end{equation}

Graphs showing the dependence of $Q^2$ vs. $x$ for various choices of the scattered lepton energy and pseudorapidity are shown in the left plots of Figs.~\ref{fig:MuICkinematics}, \ref{fig:MuIC2kinematics}, and \ref{fig:LHmuCkinematics} for different machine configurations. 

We can invert Eq.~(\ref{eq:q2eep}) to find $E^\prime_\ell$ in terms of $Q^2$ and $x$:
%
\begin{equation}
E^\prime_\ell  = E_\ell\, \bigg[ 1 - {Q^2 \over 4 E_\ell}\bigg({1 \over x E_{\rm h}} - {1\over E_\ell} \bigg)\bigg]
\label{eq:ep}
\end{equation}
%
and also invert Eq.~(\ref{eq:q2etheta}) to obtain $\theta$ (or $\eta$) in terms of $Q^2$ and $x$:
%
\begin{equation}
e^{-\eta} = \tan {\theta \over 2}   = \sqrt{ {4 E_\ell^2 \over Q^2} - {E_\ell \over x E_{\rm h}}}
\label{eq:theta}
\end{equation}
%
When the second term in the radical is negligible (for all but the smallest $x$ values), the scattered lepton angle depends only on $Q^2$:

\begin{equation}
\eta \approx -\ln \bigg( {2 E_\ell \over \sqrt{Q^2}} \bigg)
\end{equation}
%



\subsection{Hadron variables}

The DIS variables using the scattered hadronic variables are given by:
%
\begin{equation}
Q^2 = {E^{\prime 2}_{\rm h} \sin^2\gamma \over 1-y}
\label{eq:Q2h}
\end{equation}
\begin{equation}
y = {E^\prime_{\rm h} \over 2 E_\ell} \, (1-\cos\gamma)
\label{eq:yh}
\end{equation}
%
These variables
can be found from the measured hadronic energy flow via the Jacquet-Blondel
method:
%
\begin{equation}
Q^2 = {(\sum_i p_{{\rm T},i})^2 \over 1-y}
  \label{eq:jbmethq}
\end{equation}
\begin{equation}
y = {\sum_i (E_{{\rm h},i}-p_{z,i}) \over 2 E_\ell}
  \label{eq:jbmethy}
\end{equation}
%
In other words,
%
\begin{equation}
  \sin\gamma = {\sum_i p_{{\rm T},i} \over E^\prime_{\rm h}}
  \end{equation} 
%

The variable $x$ can be found from Eq.(\ref{eq:q2xy}). 

An expression for $Q^2$ as a function of $x$ based on the scattered
hadron energy $E^\prime_{\rm h}$ can be found by solving for $\cos\gamma$
in Eq.(\ref{eq:yh}) and plugging into Eq.(\ref{eq:Q2h}), expanding out
$\sin^2\gamma = (1+\cos\gamma)(1-\cos\gamma)$,
then using
Eq.(\ref{eq:q2xy}):
%
\begin{equation}
  Q^2 = {sx\,(1-{E^\prime_{\rm h} \over x\,E_{\rm h}}) \over 
      (1-{E_\ell \over x\,E_{\rm h}})}
\label{eq:q2hp}
\end{equation}
%
Likewise, an expression for $Q^2$ as a function of $x$ based on the
scattered hadronic angle $\gamma$ can be found by solving for $E^\prime_{\rm h}$ in
Eq.(\ref{eq:yh}) and plugging into plugging into Eq.(\ref{eq:Q2h}), then using
Eq.(\ref{eq:q2xy}):
%
\begin{equation}
Q^2 = {sx \over {1 + {4E^2_\ell \over sx} }{(1+\cos\gamma)
    \over (1-\cos\gamma) } } = {sx \over {1 + {E_\ell \over x\, E_{\rm h}} \cot^2({\gamma \over 2}) } }
\label{eq:q2hgamma}
\end{equation}

Graphs showing the dependence of $Q^2$ vs. $x$ for various choices of the scattered hadron energy and pseudorapidity are shown in the right plots of Figs.~\ref{fig:MuICkinematics}, \ref{fig:MuIC2kinematics}, and \ref{fig:LHmuCkinematics} for different machine configurations. 



We can invert Eq.~(\ref{eq:q2hp}) to find $E^\prime_{\rm h}$ in terms of $Q^2$ and $x$:
%
\begin{equation}
E^\prime_{\rm h}  = xE_{\rm h} + {Q^2 \over 4 E_\ell}\bigg({E_\ell \over x E_{\rm h}} - 1 \bigg)
\label{eq:hp}
\end{equation}
%
which for $x \ll E_\ell/E_{\rm h}$ is approximated by:
%
\begin{equation}
E^\prime_{\rm h}  = yE_\ell 
\end{equation}
%
We can also invert Eq.~(\ref{eq:q2hgamma}) to obtain $\gamma$ (or $\eta$) in terms of $Q^2$ and $x$:
%
\begin{equation}
e^{-\eta} = \tan {\gamma \over 2}   = { 1 \over \sqrt{{4E_{\rm h}^2 x^2 \over Q^2} - x {E_{\rm h} \over E_\ell} }  }  
\label{eq:gamma}
\end{equation}
%
which again for $x \ll E_\ell/E_{\rm h}$ can approximated by:
%
\begin{equation}
\eta \approx -\ln \bigg( {\sqrt{Q^2} \over 2 E_{\rm h}\, x } \bigg)
\end{equation}
%

\subsubsection{$\Sigma$ and $T$ variables}

A variation of the hadron method is to define the $\Sigma$ and $T$
variables:

\begin{equation}
\Sigma = \sum_i (E_{{\rm h},i}-p_{z,i}) =  2 E_\ell\; y
  \label{eq:sigma}
\end{equation}
%
\begin{equation}
T = \sqrt{  \big(\sum_i p_{x,i})^2 + (\sum_i p_{y,i} \big)^2  } = E^\prime_{\rm h} \sin\gamma
  \label{eq:T}
\end{equation}
%
which again are sums over the  hadron energies and momenta. The
variable $T$ is essentially the transverse momentum ($p_{\rm T}$) of
the hadron system (i.e. jet from the struck parton). From these definitions
we can find $Q^2$ vs. $x$ in terms of $\Sigma$ and $T$:
%
\begin{equation}
Q^2 = 2 E_{\rm h} \, x\, \Sigma
  \label{eq:Q2Sigma}
\end{equation}
%
and using the quadratic equation:
%
\begin{equation}
Q^2 = {s x \over 2} \bigg(1 \pm \sqrt{1-{4T^2\over sx}}  \bigg)
  \label{eq:Q2T}
\end{equation}
%
Graphs showing the dependence of $Q^2$ vs. $x$ for various choices of
$\Sigma$ and $T$ are shown in Fig.~\ref{fig:MuICQ2SigmaT}.
Since it can be shown that:
%
\begin{equation}
T^2 =  (1-y)\,Q^2 
\end{equation}
%
$T \approx Q$ for small $y$.


\subsection{Kinematic Resolutions}

\subsubsection{$Q^2$, Lepton Variables}

In the far backward region, $\theta \approx \pi$, we can express the relation for $Q^2$ in terms of the lepton angle from the $-z$ axis, where $\alpha = \pi - \theta$:
%
\begin{equation}
Q^2 = 2 E_\ell E^\prime_\ell\, (1- \cos \alpha)
\label{eq:Q2alpha}
\end{equation}
%
From this relation, the relative resolution on $Q^2$ in terms of the relative resolution on the measured scattered lepton energy $E^\prime$, assuming negligible uncertainty on the scattered angle, is given by:
%
\begin{equation}
  {dQ^2 \over Q^2} = {dE^\prime_{\ell} \over E^\prime_{\ell}}
\label{eq:dQ2Elep}
\end{equation}
%
The dependence of the relative uncertainty on $Q^2$ on the uncertainty of the measured scattering angle for small $\alpha$ ($\theta$ near $\pi$) can be found from $\cos\alpha \approx 1-\alpha^2/2$, such that:
%
\begin{equation}
Q^2 \approx E\,E^\prime_{\ell}\, \alpha^2
\end{equation}
This implies:
%
\begin{equation}
  {dQ^2 \over Q^2} = 2{d\alpha \over \alpha}
\label{eq:dQ2alpha}
\end{equation}
%

\subsubsection{$y$, Lepton Variables}

We can express the relation for $y$ in terms of the lepton angle from the $-z$ axis, where $\alpha = \pi - \theta$:
%
\begin{equation}
y = 1 - {E^\prime_\ell \over 2 E_\ell} \, (1+\cos\alpha)
\label{eq:Yalpha}
\end{equation}
%
We can then express the relative uncertainty on $(1-y)$ in terms of the relative resolution on the measured scattered lepton energy $E^\prime$ as:
%
\begin{equation}
{d(1-y) \over 1-y} = {dE^\prime_{\ell} \over E^\prime_{\ell}}
\end{equation}
%
Or put another way:
%
\begin{equation}
dy = {dE^\prime_{\ell} \over E^\prime_{\ell}}(1-y)
\end{equation}
%
This implies that the reconstruction of $y$ from the scattered lepton energy is good for large $y$, but degrades and reaches 100\% uncertainty for $y \approx dE^\prime_{\ell}/E^\prime_{\ell}$. 

For the dependence of the uncertainty on $1-y$ from the uncertainty on the measured scattering angle $\alpha$  we get:
%
\begin{equation}
{d(1-y) \over 1-y} \approx {\alpha\, d\alpha \over (2-\alpha^2/2)} \approx {1 \over 2} \alpha\, d\alpha 
\end{equation}
%


\subsubsection{$y$, Hadron Variables}

From Eq.(\ref{eq:yh}) we can then express the relative uncertainty on $y$ in terms of the relative resolution on the measured scattered hadron energy $E^\prime_{\rm h}$ as:
%
\begin{equation}
{dy \over y} = {dE^\prime_{\rm h} \over E^\prime_{\rm h}}
\end{equation}
%
For small angles from the far backward region, $\gamma \approx \pi$, we can express the relation for $y$ in terms of the hadron angle from the $-z$ axis, where $\beta = \pi - \gamma$:
%
\begin{equation}
y = {E^\prime_{\rm h} \over 2 E_\ell} \, (1+\cos\beta) \approx {E^\prime_{\rm h} \over 2 E_\ell} \, (2-\beta^2/2)
\end{equation}
%
The dependence of the relative uncertainty on $y$ on the uncertainty of the measured scattering angle for small $\beta$ is therefore:
%
\begin{equation}
{dy \over y} = {\beta\, d\beta \over (2-\beta^2/2)}
\end{equation}
%

%

\begin{figure}[!htb]
    \centering
    \includegraphics[width=0.43\textwidth]{MuIC-lepton.png}
    \includegraphics[width=0.45\textwidth]{MuIC-hadron.png}
    \caption{Kinematics of the scattered muon (left) and final-state hadrons (right) in the $Q^2$-$x$ plane for muon-proton deep inelastic scattering with a 1000~GeV muon beam colliding with a 275~GeV proton beam. The dashed blue lines correspond to constant  energy in GeV and the dashed red lines to constant pseudorapidity, respectively. The lower diagonal solid line corresponds to the inelasticity $y=0.01$.}
    \label{fig:MuICkinematics}
\end{figure}

\begin{figure}[!htb]
    \centering
    \includegraphics[width=0.43\textwidth]{MuIC2-lepton.png}
    \includegraphics[width=0.45\textwidth]{MuIC2-hadron.png}
    \caption{Kinematics of the scattered muon (left) and final-state hadrons (right) in the $Q^2$-$x$ plane for muon-proton deep inelastic scattering with a 1000~GeV muon beam colliding with a 1000~GeV proton beam. The dashed blue lines correspond to constant  energy in GeV and the dashed red lines to constant pseudorapidity, respectively. The lower diagonal solid line corresponds to the inelasticity $y=0.01$.}
    \label{fig:MuIC2kinematics}
\end{figure}

\begin{figure}[!htb]
    \centering
    \includegraphics[width=0.43\textwidth]{LHmuC-lepton.png}
    \includegraphics[width=0.45\textwidth]{LHmuC-hadron.png}
    \caption{Kinematics of the scattered muon (left) and final-state hadrons (right) in the $Q^2$-$x$ plane for muon-proton deep inelastic scattering with a 1500~GeV muon beam colliding with a 7000~GeV proton beam. The dashed blue lines correspond to constant  energy in GeV and the dashed red lines to constant pseudorapidity, respectively. The lower diagonal solid line corresponds to the inelasticity $y=0.01$.}
    \label{fig:LHmuCkinematics}
\end{figure}

\begin{figure}[!htb]
    \centering
    \includegraphics[width=0.43\textwidth]{MuIC-Sigma.png}
    \includegraphics[width=0.45\textwidth]{MuIC-T.png}
    \caption{$\Sigma$ (left)  and $T$ (right) in the $Q^2$-$x$ plane
      for muon-proton deep inelastic scattering with a 1000~GeV muon
      beam colliding with a 275~GeV proton beam.}
      \label{fig:MuICQ2SigmaT}
\end{figure}



\end{document}


